\chapter{Induktives Lernen}
Eine große Menge an Beispielen wird gegeben. Der Lerner muss selbst das Konzept herausfinden.

\paragraph{Version Space} \mbox{} \\
Der Raum aller Hypothesen, welche mit den Trainigsbeispielen konsistent sind.

\paragraph{Konzept} \mbox{} \\
Ein Konzept beschreibt Untermenge von Objekten oder Ereignissen definiert auf
einer größeren Menge

\paragraph{Konsistenz} \mbox{} \\
Keine negativen Beispiele werden positiv klassifiziert.

\paragraph{Vollständigkeit} \mbox{} \\
Alle positiven Beispiele werden als positiv klassifiziert.

\paragraph{Lernen als Suche im Hyphotesenraum} \mbox{}
\begin{itemize}
    \item \textbf{Algorithmen}
    \begin{itemize}
        \item Suche vom Allgemeinen zum Speziellen: Negative Beispiele führen zur
        Spezialisierung
        \item Suche vom Speziellen zum Allgemeinen: Positive Beispiele führen zur
        Verallgemeinerung
        \item Version Space: Paralleles Verwenden der oben genannten Anwendungen.
    \end{itemize}
    \item \textbf{Präzedensgraph}
    \begin{itemize}
        \item In welcher Reihenfolge werden Aktionen ausgeführt?
        \item Codierung der Hypothese mit Gerichtetem, azyklischem Graphen
    \end{itemize}
\end{itemize}

\paragraph{Version Space Algorithmus} \mbox{} \\
Der Version Space Algorithmus ist ein binärer Klassifikator für diskrete
Feature-Spaces.
\begin{itemize}
    \item Beginne mit generellster Hypothese $G=(?,...,?)$ und speziellester
    Hypothese $S = (\#,...,\#)$
    \item Ist Beispiel false
    \begin{itemize}
        \item Lösche aus S Hypothesen, die Beispiel abdecken
        \item Spezialisiere Hypothesen in G soweit, dass sie das Beispiel nicht abdecken
        und allgemeiner als eine Hypothese in S bleiben.
        \item Lösche aus G alle Hypothesen, die spezifischer als eine andere
        Hypothese in G sind
        \item Kurz: spezialisiere generellste Hypothese
    \end{itemize}
    \item Ist Beispiel true
    \begin{itemize}
        \item Lösche aus G Hypothesen, die mit Beispiel inkonsistenten Hypothesen
        \item Verallgemeinere die Hypothesen in S soweit, dass sie Beispiel abdecken
        und dass sie spezifischer als eine Hypothese in G bleiben
        \item Lösche aus S alle Hypothesen, die allgemeiner als eine andere
        Hypothese aus S sind
        \item Kurz: Passe speziellste Hypothese an und verallgemeinere
    \end{itemize}
    So kann man den Raum aller mit den Trainigsdaten konsistenten Hypothesen
    finden.

    \item Positive Aspekte
    \begin{itemize}
        \item feststellbar, welche Art von Beispiele noch nötig ist
        \item feststellbar, wann das Lernen abgeschlossen ist
    \end{itemize}
\end{itemize}
\paragraph{Induktiver Bias} \mbox{} \\
Induktives Lernen erfordert Vorannahmen. Bias ist die Vorschrift, nach der Hypothesen
gebildet werden.
