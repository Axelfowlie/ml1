\chapter{Evolutionäre Algorithmen}

\mparagraph{Nomenklatur}
\begin{itemize}
    \item \textbf{Individuum}: eine mögliche Lösung, Hypothese
    \item \textbf{Population und Generation}: Lösung bzw. Hypothesenmenge
    \item \textbf{Erzeugen von Nachkommen}: Generierung neuer Hypothesen
    Methoden z.B \begin{itemize}
        \item Rekombination: Vermischen der Gene zweier Elternteile.
        \begin{itemize}
            \item diskret: ein Teil Gene von einem Elternteil
            übernommen werden und andere vom anderen Elternteil.
            \item intermediär: Nachkomme wird gemittelt $z_i = \frac{(x_i + y_i)}{2}$
            \item Crossover: aus 2 Eltern werden 2 Nachkommen. singlepoint,
            twopoint, uniform crossover.
        \end{itemize}
        \item Mutation: Veränderung einzelner Gene bei Abstammung von einem Elternteil.
        z.B
        \begin{itemize}
            \item Bitinversion: Feste anzahl aber zufällige Gene
            \item Translation: Verschieben einer Teilsequenz
            \item Invertiertes Einfügen
            \item spezielle Mutationsoperatoren sind andwendungsspezifisch.
        \end{itemize}
    \end{itemize}
    \item \textbf{Veränderter Nachfolger, Kind, Nachkomme}: neue Hypothese
    \item \textbf{Fitness Funktion}: Hypothesengüte, zu optimierendes Kriterium
    \item \textbf{Selektion der Besten}: Auswahl der Hypothesen, welche die
    beste Problemlösung erzeugen.
\end{itemize}

\mparagraph{Grundalgorithmus}
Solange fitness nicht optimal mache:
\begin{enumerate}
    \item Selektion der Eltern
    \item Generierung von Nachkommen
    \item Fitness bewertung
    \item Selektion der überlebenden Populationsmitglieder
\end{enumerate}

\mparagraph{Selektion}

Zwei arten der Selektion:
\begin{itemize}
    \item die Eltern für jeweilige Erzeugung von Nachkommen (mating)
    \item der Population für die nächste Iteration
\end{itemize}

Probleme:
\begin{itemize}
    \item \textbf{Genetische Drift}: Individuen vermehren sich zufällig mehr
    als andere. Diese sind nicht unbedingt besser für das Problem geeignet.
    \item \textbf{Crowding, Ausreißerproblem}: ''fitte'' Individuen und ähnliche
    Nachkommen dominieren die Population. Problematisch wegen lokaler Maxima.
\end{itemize}


\mparagraph{Mating}
Ist ein Populationsmodell
\begin{itemize}
    \item \textbf{Insellmodell}: lokal, die Evolution läuft weitgehend getrennt,
    nur manchmal werden Individuen ausgetauscht.
    \item \textbf{Nachbarschaftsmodell}: nahe Umgebung, Nachkommen dürfen nur von
    Individuen gezeugt werden, die in ihrere Nachbarschaft die beste Fitness
    besitzen.
    \item \textbf{Eine einfache Menge}: global, die global Besten entwickeln
    sich rasch weiter, andere Entwicklungslinien werden unterdrückt.
\end{itemize}

\mparagraph{Selektionsmethoden}

\begin{itemize}
    \item Fitness Based Selection \\
    $P(x) \approx \frac{\lambda}{\mu}\frac{f(x)}{\sum_{x' \in \text{Pop}} f(x')}$

    \begin{itemize}
        \item $P(x)$: Wahrschl. der Auswahl von Individuum $x$
        \item $\lambda$ Anzahl von Nachkommen
        \item $\mu$ Populationsgröße
        \item $f$ Fitness Funktion
    \end{itemize}
    \item Ranking Based Selection \\
    $P(x) \approx \frac{g(r(x))}{\sum_{x' \in \text{Pop}} g(r(x'))}$

    \begin{itemize}
        \item $P(x)$: Wahrschl. der Auswahl von Individuum $x$
        \item $r(x)$ ranking von x in der aktuellen Population gemäß Fitness
        Funktion
        \item $g$ Mit der Güte des Ranges monoton steigende Funktion
    \end{itemize}
    \item Tournament Selection
\end{itemize}

\mparagraph{Evolution}
\begin{itemize}
    \item \textbf{Lamarksche Evolution}: Individuen ändern sich (lernen) nach
    der Erzeugung. Genotyp (alle Gene) wird verändert und anschließend vererbt.
    \item \textbf{Baldwinsche Evolution}: Individuen ändern sich (lernen) nach
    der Erzeugung. Genotyp bleibt unverändert
    \item \textbf{Hybride Verfahren}: Es gibt veränderbare und fixe Phänotypen.
\end{itemize}

\mparagraph{Anwendungsbeispiele}
\begin{itemize}
    \item Travelling Salesman
    \item Flugplanoptimierung
    \item Mischung von Kaffeesorten
    \item Cybermotte: Motte müssen optimales Muster finden, um sich vor einer
    Fläche weißem Rauschen zu verbergen
    \item Optimale Steuerung in der Robotik
    \item Optimierung der Topologie Neuronaler Netze
    \item Optimierung des Muskel-Skelett Systems hinsichtlich energieeffizienter
Steuerung


\end{itemize}
