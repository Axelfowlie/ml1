\chapter{Deduktives Lernen}
Fakten werden gegeben. Der lernende bekommt das allgemeine Konzept gesagt und
 muss nur logische Schlussfolgerungen machen.

\mparagraph{Erklärungsbasiertes Lernen, EBL Explanation Based Learning}
The key insight behind explanation-based generalization
is that it is possible to form a justified generalization of a
single positive training example provided the learning
system is endowed with some \textbf{explanatory capabilities}. In
particular, the system must be able to explain to itself \textbf{why
the training example is an example of the concept} under
study. Thus, the generalizer is presumed to possess a
definition of the concept under study as well as domain
knowledge for constructing the required explanation.\\

Gegeben:
\begin{itemize}
    \item Zielkonzept: Beschreibung des zu lernenden Konzepts
    \item Trainigsbeispiel: Beispiel für das Zielkonzept
    \item Bereichstheorie: Regeln und Fakten, die erklären warum Trainigsbeispiel
    ein Beispiel für das Zielkonzept ist
    \item Operationalitäts-(Anwendbarkeits) Kriterium: Ein Prädikat über
    Konzeptbeschreibung, das die Form spezifiziert, in der erlernte Beschreibungen
    vorliegen müssen.
\end{itemize}

Gesucht:
\begin{itemize}
    \item Eine Generalisierung des Trainigsbeispiels, die eine hinreichende
    Definition des Zielkonzeptes darstellt und das Operationalitätskriterium erfüllt.

\end{itemize}
\mparagraph{Explanation Based Generalization, EBG}
Prozess, der implizites Wissen in explizites Wissen umwandelt. \\
EBG ist ein Zweischrittverfahren und geht wie folgt vor:
\begin{itemize}
    \item Explain: Finden einer Erklärung, die zeigt, warum das Trainigsbeispiel
    die Definition des Zielkonzeptes erfüllt $\rightarrow$ Modus Ponens anwenden.
    \item Generalize: Bestimme hinreichende Bedingungen, unter denen die oben
gefundene Erklärungsstruktur gültig ist und formuliere diese
Kriterien in Termen, die das Operationalitätskriterium erfüllen.
\end{itemize}
Bei EBG werden Makrooperatoren (Zusammengefasste Operatorsequenzen, welche die
Kosten für das Finden von Problemlösungswissen reduzieren) erzeugt.
STRIPS verwendet EGB.

\mparagraph{Deduktives vs. Induktives lernen}

\begin{table}[h!]
\centering

\begin{tabular}{|l|l|l|}
\hline
 & \textbf{Induktives Lernen} & \textbf{Deduktives Lernen} \\ \hline
\textbf{Ziel} & Hypothese passt du den Daten & Hypothese passt zur Bereichstheorie \\ \hline
\textbf{Vorteile} & wenig a-priori wissen & wenig Beispiele notwendig \\ \hline
\textbf{Nachteile} & \begin{tabular}[c]{@{}l@{}}- schlecht bei geringen Datenmengen\\ - schlecht bei inkorrektem Bias\end{tabular} & schlecht, falls imperfekte Bereichstheorie \\ \hline
\end{tabular}
\end{table}

\mparagraph{Lernproblem}
Seien $D$ Trainigsbeispiele, möglicherweise mit Fehlern, $B$ die Bereichstheorie,
möglicherweise Fehlerhaft und $H$ der Hypothesenraum. So lässt sich die
beste Hypothese $h$, welche am besten zu Trainigsbesiepiel als auch Bereichstheorie
passt mittels
\begin{displaymath}
    \argmin_{h \in H} k_D E_D(h) + k_bE_B(h)
\end{displaymath}
berechnen. Wobei
\begin{itemize}
    \item $E_D, E_B$: Fehlerrate bezüglich Trainigsdaten/Bereichstheorie
    \item $k_D, k_B$: relatives Gewicht für Trainigsdaten/Bereichstheorie
\end{itemize}

\mparagraph{Knowledge Based Artificial Neurol Network, KBANN}
KBANN ist ein hybrides Verfahren. Die Idee hierbei ist es mittels Bereichstheorie
ein Neuronales Netz zu initialisieren, welches durch Backpropagation und
Trainigsbeispiele verfeinert wird.

\begin{enumerate}
    \item Pro Instanzattribut wird ein Netz-Input verwendet. Für jede Klausel
    wird ein Neuron hinzugefügt
    \item Dieses ist mit dem Instanzattribut durch das Gewicht $w$ verbunden
    wenn es nicht negiert ist. Ansonsten $-w$
    \item Der Schwellwert wird auf $-(n-0.5)w$ gesetzt. $n$ ist die Anzahl der
    nicht negierten Bedingungsteile
    \item Verbinde die restlichen Neuronen von Schicht $i$ mit Schicht $i+1$ indem
    zufällige kleine Gewichte gesetzt werden.
\end{enumerate}

Anwendung:
\begin{itemize}
    \item Lernen von physikalischen Objektenklassen
    \item Erkennung biologischer Konzepte in DNS-Sequenzen
\end{itemize}
