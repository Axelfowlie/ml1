\chapter{Reinforcement Learning}
Lernen durch Belohnung \\
\textbf{Lernziel:} Finde Aktionssequenz $a_1, a_2, ...,a_n$, sodass dadurch die
maximale Bewertung aufgesammelt wird

\paragraph{Markrow-Entscheidungsproblem (Markrov Decision Process, MDP)} \mbox{}\\

Ein MDP ist ein 5-Tupel $(S, A, P, R, \gamma)$ wobei
\begin{itemize}
    \item $S$: Endliche Zustandsmenge (States), ''Sensorik''
    \item $A(s)$: Menge möglicher Aktionen im Zustand $s$, ''Aktorik''
    \item Zustandsänderung (meist bekannt oder beobachtbar )\\
    $\delta: (S \times A) \rightarrow S$\\ $\delta(s_t,a_t) = s_{t+1}$\\
    Markov Bedingung: keine Abhängigkeit von der Vergangenheit
    \item Bewertung von Aktionen (direkt oder indirekt messbar/bekannt)\\
    $r:(S \times A) \rightarrow R \\ r(s_t,a_t) = r_t \\$
    Die direkte Belohnung wenn durch Aktion a im Zustand s ausgeführt wurde
    \item $\gamma \in [0,1]$ Diskontierungsfaktor, zeigt Bedeutung von direkter
    Belohnung im Vergleich zu zukünfitger Belohnung an.
    \begin{itemize}
        \item $\gamma = 0$: aktuelle Aktionsbewertung ist wichtig (1-Step)
        \item $\gamma > 0$: zukünftige (letzte) Bewertungen werden berücksichtigt. (n-step)
        \item $\gamma < 1$: Notwendig um konvergente V-Funktionen (Value function) zu erhalten.
    \end{itemize}
\end{itemize}

\section{Reinforcement Learning}
Beim RL liegt ein MDP vor. Es gibt also einen Agenten, der Aktionen ausführen kann.
Diese können (nicht notwendigerweise) sofort bewerten werden.

\paragraph{Policy} \mbox{} \\
Policy ist eine Vorschrift $\pi: S \rightarrow A$, in welchem Zustand welche Aktion
ausgeführt werden soll.
\paragraph{Policy Learning} \mbox{} \\
Gesucht $S_{1} \xrightarrow[a_1]{r_1}S_{2}\xrightarrow[a_2]{r_2}\dots \xrightarrow[a_{n-1}]{r_{n-1}}S_n$
Finde optimale Policy $\pi^{*}$
\paragraph{Value Function} \mbox{} \\
Die Funktion $V^{\pi}:S\rightarrow \mathbb{R}$ gibt den erwarteten Wert (nicht Belohnung, da $\gamma$ eingeht)
eines Zustand $s$ unter der policy $\pi$ an. Mit $V^{*}$ wird der Wert unter der
optimalen policy bezeichnet. \\
$V^{\pi}(s_t) = r_t + {\gamma} r_{t+1} + {\gamma}^2 r_{t_2} + \dots = \sum\limits_{i=0}^{\infty} \gamma^i r_{t+1}$
\begin{itemize}
    \item optimale Zielfunktion: $\pi^{*}(s) = \underset{\pi}{\mathrm{argmax}} V^{\pi}, \forall s$
    \item max. akkumulierte Bewertung: $V^{*} = V^{\pi^{*}}(s)$
    \item rekursive Definition (Bellman Gl.): $V^{*}(s_t) = r_t + \gamma V^{*}(s_{t+1})$
\end{itemize}

\paragraph{Simple Value Iteration} \mbox{} \\
\begin{algorithm}[H]
    \begin{algorithmic}
        \State Initialisiere $\hat{V^{*}}(s)$ zufällig
        \While{} forever
            \State wähle einen Zustand $S_t$
            \State ermittle beste Aktion und den Folgezustand
            \State $s_t = \delta(s_t,\pi^{*}(s_t)))$
            \State ersetzte
            \State $V^{*}(s_t) = r_t + \gamma V^{*}(s_{t+1})$
            \EndWhile
    \caption{Value Iteration}
    \end{algorithmic}
\end{algorithm}

Simple Value Iteration schätzt die Value Function ab, indem sie solange wie nötig
aktualisiert wird bis sie konvergiert.

\paragraph{Simple Temporal Difference Learning} \mbox{} \\
Funktioniert genau wie Simple Value Iteration, nur dass die
Value Function mit einer learning rate $\alpha$ aktualisiert wird:\\
\begin{itemize}
    \item $\hat{V}^{*}(s_t) = (1-\alpha) * \hat{V}^{*}(s_t)+\alpha(r_t + \gamma \hat{V}^{*}(s_{t+1}))$
\end{itemize}
\paragraph{Q-Funktion} \mbox{} \\
Die Funktion $Q : S \times A \rightarrow \mathbb{R}$, $Q(s,a)$ gibt die maximale
Bewertung, die erreicht werden kann im Zustand $s$ durch die Aktion $a$.
\paragraph{Q-Learning} \mbox{} \\
\begin{algorithm}
\begin{algorithmic}
    \State Ziel: Finde Schätzung $\hat{Q}(s,a)$ der absoluten Funktion $Q(s,a)$
    \State Lernen
    \State Initialisiere $\forall s,a$ $\hat{Q}(s,a) = 0$
    \State Wähle Zustand
    \While{} Forever:
        \State wähle Aktion $a$ und führe aus
        \State $r \leftarrow$ direkte Bewertung (reward)
        \State neuer Zustand $s'$
        \State update
        \IndState $\hat{Q}(s,a) \leftarrow r + \gamma \max_{a'} \hat{Q}(s',a')$
        \State $s\leftarrow s'$
        \EndWhile
    \caption{Q-Learning}
\end{algorithmic}

\end{algorithm}

\paragraph{Temporal Difference Learning}\mbox {} \\
Die Differenz folgender Schätzungen als Lernsignal für $\hat{Q}(s,a), \hat{V}^{*}(S)$

Zum predicten nutzt TD die Annahme, dass aufeinander folgende Vorhersagen
oftmals in gewissen Maßen zusammenhängen. Schätzungen werden anhand anderer
gerlernter Schätzungen angepasst ohne auf das Endgültige Ergebniss zu warten.

\mparagraph{Vorwärtssicht (theoretisch)}
\begin{itemize}
    \item Gewichtete Anpassung an direkt nachfolgender Schätzung (1-Step) oder
    \item Gewichtete Anpassung an n Schritte nachfolgender Schätzung (n-step)
\end{itemize}

\mparagraph{Rückwärtssicht}
Fehlersignale (temporal differences) in den Schätzungen werden nach hinten
weitergegeben.

\paragraph{Eligibility Traces, Verantwortlichkeitsspuren} \mbox{} \\
Eligibility Traces sind ein Grundmechanismus des Reinforcment Learning. \\
Es gibt 2 Arten diese zu betrachten:
\begin{itemize}
    \item Die theoretische Weise, dass sie die Brücke von Temporal Difference
    zur Monte Carlo Methode sind (Vorwärtssicht)
    \item Oder, dass eligiblity traces eine Art temporäre Aufnahme eines
    Geschehens in einem Event sind (z.B das Erreichen eines Zustandes oder das
    Ausführen einer Aktion (Rückwärtssicht). Eligibility Traces bietet also ein
    Kurzzeitgedächnis von vielen vorheriger Inputs so dass neue Beobachtungen
    in Bezug auf diese Signale aktualisiert werde können.
\end{itemize}
Die Verwendung von Eligibility Traces wird durch $(\lambda)$ gekennzeichnet (z.B
$TD(\lambda)$)
\paragraph{SARSA} \mbox{} \\
Ist ein Lernalgorithmus, der die Q-Funktion updated:
$Q(s_t,a_t) \leftarrow (1-\alpha) \cdot Q(s_t,a_t) + \alpha [r_{t+1} + \gamma Q(s_{t+1}, a_{t+1})]$
