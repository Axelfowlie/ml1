\chapter{Lerntheorie}

\mparagraph{Rasiermesser Prinzip}
Löse nie ein Problem komplizierter als nötig, denn die einfachste, richtige Erklärung
ist die Beste.\\
Von mehreren möglichen Erklärungen für ein und denselben Sachverhalt ist die einfachste
Theorie allen anderen vorzuziehen. Eine Theorie ist einfach, wenn sie möglichst
wenige Variablen und Hypothesen enthält und wenn diese in klaren logischen
Beziehungen zueinander stehen, aus denen der zu erklärende Sachverhalt logisch folgt.

\mparagraph{Definition Lernmaschine}
\begin{itemize}
    \item Eine lernende Maschine wird bestimmt durch:
    \begin{itemize}
        \item Hypothesenraum ${h_\alpha : \alpha \in A}$
        \item Lernverfahren: die Methode um $\alpha_{opt}$ mit Hilfe von
        Lernbeispielen zu finden
    \end{itemize}
    \item Das resultierende Entscheidungsmodell $M_{opt}$. Es ist gegeben durch die Auswertung
    der optimalen Hypothese $h_{opt}$, die durch die lernende(n) Maschine(n) bestimmt wird
\end{itemize}

\mparagraph{Probleme}

\begin{itemize}
    \item \textbf{Statistisches Problem}: Das Verfahren betrachtet einen -
    gemessen an der Menge der Trainigsdaten - ''zu grossen'' Hypothesenraum
    \item \textbf{Komplexitätsproblem}: Aufgrund der Komplexität des Problems kann
    das Lernverfahren nicht das Finden einer optimalen Lösung innerhalb des
    Hypothesenraum garantieren.
    \item \textbf{Repräsentationsproblem}: Der Hypothesenraum enthält keine ausreichend
    gute Approximation der Zielfunktion/Konzept etc.
\end{itemize}

\mparagraph{Overfitting}
Die Tendenz der Maschine sich beim Lernen auf die Lernbeispiele zu spezialisieren
(Auswendig lernen). $\rightarrow$ Lernfehler fällt, Testfehler steigt, Generalisierung fällt.\\

\textbf{Lösung}
\begin{itemize}
    \item Repräsentative Beispiele
    \item Lernprozess durch den Verifikationsfehler steuern.
    \item Richtige Wahl und Suche der optimalen Hypothese $h_a$
\end{itemize}


\mparagraph{Validierung}
\begin{itemize}
    \item Crossvalidation
    \begin{itemize}
        \item Teile Daten in Lern und Validierungsdaten
        \item Bestimmte darauf verschiedene Hypothesen (bzw. deren Parameter)
        \item Berechne jeweils Generalisierung
        \item wiederhole
    \end{itemize}
    \item n-fold-crossvalidation
    \begin{itemize}
        \item Zerlege Daten in n-Mengen
        \item Trainiere auf n-1 Mengen, Teste auf 1 Menge
        \item Wiederhole
    \end{itemize}
    \item Leave One Out
    \begin{itemize}
        \item Jeweils ein Beispiel für das Lernen weglassen
        \item Addiere Fehler für weggelassene Beispiele
        \item Wiederhole
    \end{itemize}
\end{itemize}

\mparagraph{Boosting}
Kombiniere mehrere schwache Modelle um ein gutes zu bekommen, indem Trainigsbeispiele unterschiedlich
gewichtet werden.
\mparagraph{Bagging (Bootstrap aggregating)}
Kombiniere mehrere schwache Modelle um eine gutes zu bekommen. Dabei bekommt jedes
schwache Modell nur eine Teilmenge der Trainigsdaten

\mparagraph{AdaBoost (Adaptive Boosting)}
Lerne einen Klassifikator für die Daten. Finde die fehlerhaft klassifizierten Beispiele
, erhöhe deren Gewichte und trainieren einen neuen Klassifikator, welcher darauf achtet
die höher gewichteten Beispiele richtig zu klassifizieren

\begin{algorithm}[H]
    \begin{algorithmic}
        \State Begin: $D = {(\vec{x}_1,y_1),\dots,(\vec{x}_n,y_n)}$, $W_1(i) = \frac{1}{n}$ (Gewicht pro Beispiel)
        \For {k=1 till $k_{max}$}
        \State Trainiere $M_k$ auf $D_k$ (Anzahl $D_k = n$, Bsp. gewählt ab. von $W_k(i)$)
        \State $E_k \leftarrow$ emp. Fehler von $M_k$ (gewichtet bez. $W_k(i)$) )
        \State $\alpha_k \leftarrow \frac{1}{2} \log \frac{1-E_k}{E_k}$
        \begin{equation}W_{k+1}(i) \leftarrow \frac{W_k(i)}{Z_k}\begin{cases}
            e^{-\alpha_k}, \text{ wenn } h_k(\vec{x}_i) = y_i \\
                                                            e^{\alpha_k}, \text{wenn } h_k(\vec{x}_i) \neq y_i
                                                        \end{cases}\end{equation}
        \EndFor
        \caption{AdaBoost}
    \end{algorithmic}
\end{algorithm}

\mparagraph{Kaskadierung}
Verwende eine Kaskade an Entscheidungen anstatt kombinierter Gewichte.
Erkennt ein Teilerkenner der Kaskade ein Beispiel als Falsch an, stopp. Oder so
ähnlich.

\mparagraph{Probably approximately correct learning (PAC)}
PAC macht eine Aussage über die anzahl der benötigten Stichproben, wenn
man einen bestimmten realen Fehler mit einer frei zu wählenden Wahrscheinlichkeit
bekommen will.\\\\

Gegeben
\begin{itemize}
    \item Eine Menge $X$ von Instanzen der Länge $n$
    \item ein Konzept $C$
    \item Ein Hypothesenraum $H$
    \item eine Lerndatenmenge $D$
\end{itemize}
Es kann nun mit einer beliebigen Wahrscheinlichkeit $1-\delta, 0<\delta<\frac{1}{2}$ \\
eine $\epsilon$ - genaue Hypothese gefunden werden. $E_d(h) \leq \epsilon, 0<\epsilon<\frac{1}{2}$

Die Anzahl der benötigten Lerndaten ist: \\
$m \geq \frac{1}{\epsilon}(\ln\frac{1}{\delta}+\ln|H|)$ \\
$\Rightarrow$ je größer die gewünschte Sicherheit, je kleiner der zulässige fehler, je größer
der Hypothesenraum, desto größer die Anzahl der benötigten Daten.

\mparagraph{Vapnik-Chervonenkis Dimension}
Die VC Dimension $VC(h_a)$ von $H^\alpha$ ist gleich der maximalen Anzahl von Datenpunkten
(aus einer Menge $S$) die von $H^\alpha$ beliebig separiert werden können.
Eien Abbildung (Hypothese) $h$ separiert die Daten aus $S$ wenn durch $h$ zwei Untermengen
definiert werden:
$\{x|h(x) = 0\} \text{ und } \{x|h(x) = 1\}$

\mparagraph{Structural Risc Minimization, SRM}
SRM ist die Abwägung zwischen einem einfachen Modell und einem komplexen Modell,
welches auf den Trainingsdaten besser funktioniert, aber eventuell mehr unter
Overfitting leidet.

\mparagraph{Abschätzen des realen Fehlers}
Der reale Fehler kann durch den empirischen Fehler und die VC-Dimension wie
folgt abgeschätzt werden:

Mit Wahrscheinlichkeit $P(1-\eta)$ gilt:
\begin{displaymath}
        E(h_\alpha) \leq E_{emp}(h_\alpha) + \sqrt{\frac{VC(h_\alpha)}{N} \cdot (\log(2 N / VC(h_\alpha)) + 1) - \frac{\log(\eta  / 4)}{N}}
\end{displaymath}


wobei gilt:

\begin{itemize}
    \item $E(h_\alpha)$ ist der reale Fehler der mit der Hypothese $h_\alpha$
        gemacht wird
    \item $E_{emp}(h_\alpha)$ ist der empirische Fehler der mit der Hypothese $h_\alpha$
        gemacht wird
    \item $VC(h_\alpha)$ ist die VC-Dimension der Lernmaschine
    \item N ist die Anzahl der Lernbeispiele
    \item $0 \leq \eta \leq 1$
\end{itemize}
